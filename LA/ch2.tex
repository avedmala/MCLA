\chapter{Systems of Linear Equations}

\section{Introduction to Systems of Linear Equations}
A linear equation in the n variables $x_1, x_2, \dots , x_n$ is an equation
that can be written in the form
$$a_1x_1+a_2x_2+\dots+a_nx_n=b$$
Where the coefficients $a_1, a_2, \dots , a_n$ and the constant term $b$ are constants.

\subsection*{Example}
Solve the system
\begin{enumerate}
    \item[] $x-y-z=2$
    \item[] $y+3z=5$
    \item[] $5z=10$
\end{enumerate}

\subsection*{Solution}
Starting from the last equation and working backward, we find successively that
$z=2$, $y=5-3(2)=-1$, and $x=2+(-1)+2=3$. So the unique solution is $[3,-1,2]$.

\section{Direct Methods for Solving Linear Systems}
The \textbf{coefficient matrix} contains the coefficients of the variables, and
the \textbf{augmented matrix} is the coefficient matrix augmented by an extra column
containing the constant terms.

A matrix is in \textbf{row echelon form} if it satisfies the following properties:
\begin{enumerate}
    \item Any rows consisting entirely of zeros are at the bottom
    \item In each nonzero row, the first nonzero entry (called the leading entry)
          is in a column to the left of any leading entries below it
\end{enumerate}

\subsection*{Example}
Reduce the following matrix to echelon form:
$$\begin{bmatrix}
        1  & 2 & -4 & -4 & 5 \\
        2  & 4 & 0  & 0  & 2 \\
        2  & 3 & 2  & 1  & 5 \\
        -1 & 1 & 3  & 6  & 5
    \end{bmatrix}$$

\subsection*{Solution}

\begin{enumerate}
    \item[] $\begin{bmatrix}
                  1  & 2 & -4 & -4 & 5 \\
                  2  & 4 & 0  & 0  & 2 \\
                  2  & 3 & 2  & 1  & 5 \\
                  -1 & 1 & 3  & 6  & 5
              \end{bmatrix}
              \arrows3{R_2-2R_1}{R_3-2R_1}{R_4+R_1}
              \begin{bmatrix}
                  1 & 2  & -4 & -4 & 5  \\
                  0 & 0  & 8  & 8  & -8 \\
                  0 & -1 & 10 & 0  & -5 \\
                  0 & 3  & -1 & 2  & 10
              \end{bmatrix}$
    \item[] $\arrows1{R_2 \leftrightarrow R_3}
              \begin{bmatrix}
                  1 & 2  & -4 & -4 & 5  \\
                  0 & -1 & 10 & 9  & -5 \\
                  0 & 0  & 8  & 8  & -8 \\
                  0 & 3  & -1 & 2  & 10
              \end{bmatrix}$
    \item[] $\arrows1{R_4+3R_2}
              \begin{bmatrix}
                  1 & 2  & -4 & -4 & 5  \\
                  0 & -1 & 10 & 9  & -5 \\
                  0 & 0  & 8  & 8  & -8 \\
                  0 & 0  & 29 & 29 & 5
              \end{bmatrix}$
    \item[] $\arrows1{\frac{1}{8}R_3}
              \begin{bmatrix}
                  1 & 2  & -4 & -4 & 5  \\
                  0 & -1 & 10 & 9  & -5 \\
                  0 & 0  & 1  & 1  & -1 \\
                  0 & 3  & -1 & 2  & 10
              \end{bmatrix}$
    \item[] $\arrows1{R_4-29R_3}
              \begin{bmatrix}
                  1 & 2  & -4 & -4 & 5  \\
                  0 & -1 & 10 & 9  & -5 \\
                  0 & 0  & 1  & 1  & -1 \\
                  0 & 0  & 0  & 0  & 24
              \end{bmatrix}$
\end{enumerate}

\subsection*{Gaussian Elimination}
When row reduction is applied to the augmented matrix of a system of linear
equations, we create an equivalent system that can be solved by back substitution.
\begin{enumerate}
    \item Write the augmented matrix of the system of linear equations.
    \item Use elementary row operations to reduce the augmented matrix to row echelon form.
    \item Using back substitution, solve the equivalent system that corresponds to the row-reduced matrix.
\end{enumerate}

The \textbf{rank} of a matrix is the number of nonzero rows in its row echelon form.

\subsection*{Example}
Solve the system:
\begin{enumerate}
    \item[] $x_1-x_2+2x_3=3$
    \item[] $x_1+2x_2-x_3=-3$
    \item[] $2x_2-2x_3=1$
\end{enumerate}

\subsection*{Solution}
\begin{enumerate}
    \item[] $\begin{bmatrix}
                  1 & -1 & 2  & 3  \\
                  1 & 2  & -1 & -3 \\
                  0 & 2  & 2  & 1
              \end{bmatrix} \arrows1{R_2-R_1} \begin{bmatrix}
                  1 & -1 & 2  & 3  \\
                  0 & 3  & -3 & 6  \\
                  0 & 2  & -2 & -1
              \end{bmatrix}$
    \item[] $\arrows1{\frac{1}{3}R_2}\begin{bmatrix}
                  1 & -1 & 2  & 3  \\
                  0 & 1  & -1 & -2 \\
                  0 & 2  & -2 & 1
              \end{bmatrix}\arrows1{R_3-2R_2}\begin{bmatrix}
                  1 & -1 & 2  & 3  \\
                  0 & 1  & -1 & -2 \\
                  0 & 0  & 0  & 5
              \end{bmatrix}$
\end{enumerate}
Leading to the impossible equation $0=5$. Thus, the system has no solutions.

\subsection*{Gauss-Jordan Elimination}
Modification of Gaussian elimination greatly simplifies the back substitution phase
and is particularly helpful when calculations are being done by hand or on a system
with infinitely many solutions. This method relies on reducing the augmented matrix even further.
\begin{enumerate}
    \item Write the augmented matrix of the system of linear equations.
    \item Use elementary row operations to reduce the augmented matrix to row echelon form.
    \item If the resulting system is consistent, solve for the leading variables in terms of any remaining free variables.
\end{enumerate}

A matrix is in \textbf{reduced row echelon form} if it satisfies the following properties:
\begin{enumerate}
    \item It is in row echelon form.
    \item The leading entry in each nonzero row is a 1 (called a leading 1).
    \item Each column containing a leading 1 has zeroes everywhere else.
\end{enumerate}

\subsection*{Example}
Determine whether the lines $\*x=\*p+s\*u$ and $\*x=\*q+t\*v$ intersect and,
if so find their point of intersection when
$$\*p=\begin{bmatrix}
        1 \\ 0 \\ -1
    \end{bmatrix} \quad \*q=\begin{bmatrix}
        0 \\ 2 \\ 1
    \end{bmatrix} \quad \*u=\begin{bmatrix}
        1 \\1\\1
    \end{bmatrix} \quad \*v=\begin{bmatrix}
        3 \\ -1 \\ -1
    \end{bmatrix}$$

\subsection*{Solution}
$$\*x=\*p+s\*u=\*q+t\*v \quad\text{or}\quad s\*u-t\*v=\*q-\*p$$
\begin{enumerate}
    \item[] $s-3t=-1$
    \item[] $s+t=2$
    \item[] $s+t=2$
\end{enumerate}
From this, we find that $s=\frac{5}{4}$, $t=\frac{3}{4}$.

The point of intersection is therefore
$$\begin{bmatrix}
        x \\ y\\ z
    \end{bmatrix}=\begin{bmatrix}
        1 \\ 0 \\ -1
    \end{bmatrix}+\frac{5}{4}\begin{bmatrix}
        1 \\ 1\\ 1
    \end{bmatrix}=\begin{bmatrix}
        9/4 \\ 5/4 \\ 1/4
    \end{bmatrix}$$

A system of linear equations is called \textbf{homogeneous} if the
constant term in each equation is zero.

\section{Spanning Sets and Linear Independence}

\subsection*{Example}
Is the vector $\begin{bmatrix}
        1 \\ 2 \\ 3
    \end{bmatrix}$ a linear combination of the vectors $\begin{bmatrix}
        1 \\ 0 \\ 3
    \end{bmatrix}$ and $\begin{bmatrix}
        -1 \\ 1 \\ -3
    \end{bmatrix}$?

\subsection*{Solution}
We want to find scalars $x$ and $y$ such that
$$x\begin{bmatrix}
        1 \\ 0 \\ 3
    \end{bmatrix}+y\begin{bmatrix}
        -1 \\ 1 \\ -3
    \end{bmatrix}=\begin{bmatrix}
        1 \\ 2 \\ 3
    \end{bmatrix}$$

Expanding, we obtain the system
\begin{enumerate}
    \item[] $x-y=1$
    \item[] $y=2$
    \item[] $3x-3y=3$
\end{enumerate}

Whose augmented matrix is
$$\begin{bmatrix}
        1 & -1 & 1 \\
        0 & 1  & 2 \\
        3 & -3 & 3
    \end{bmatrix}\arrows1{ref}\begin{bmatrix}
        1 & 0 & 3 \\
        0 & 1 & 2 \\
        0 & 0 & 0
    \end{bmatrix}$$

So the solution is $x=3$, $y=2$, and the corresponding linear combination is
$$3\begin{bmatrix}
        1 \\ 0 \\ 3
    \end{bmatrix}+2\begin{bmatrix}
        -1 \\ 1 \\ 3
    \end{bmatrix}=\begin{bmatrix}
        1 \\ 2 \\ 3
    \end{bmatrix}$$

If $S=\{\*v_1,\*v_2,\dots,\*v_k\}$  is a set of vectors in $\mathbb{R}^n$, then
the set of all linear combinations of $\*v_1,\*v_2,\dots,\*v_k$ is called
the \textbf{}{span} of $\*v_1,\*v_2,\dots,\*v_k$ and is denoted by span($\*v_1,\*v_2,\dots,\*v_k$)
or span($S$). If span($S$) = $\mathbb{R}^n$, then $S$ is called a
\textbf{spanning set} for $\mathbb{R}^n$.

\subsection*{Example}
Show that
$$\mathbb{R}^2=\text{span}\left(\begin{bmatrix}
            2 \\ -1
        \end{bmatrix},\begin{bmatrix}
            1 \\ 3
        \end{bmatrix}\right)$$

\subsection*{Solution}
We need to show that an arbitrary vector $\begin{bmatrix}
        a \\ b
    \end{bmatrix}$ can be written as a linear combination of $\begin{bmatrix}
        2 \\ -1
    \end{bmatrix}$ and $\begin{bmatrix}
        1 \\ 3
    \end{bmatrix}$.

Using the augmented matrix and row reduction we produce:
$$\begin{bmatrix}
        2  & 1 & a \\
        -1 & 3 & b
    \end{bmatrix}\arrows1{R_1\leftrightarrow R_2}\begin{bmatrix}
        -1 & 3 & b \\
        2  & 1 & a
    \end{bmatrix}\arrows1{R_2+2R_1}\begin{bmatrix}
        -1 & 0 & b    \\
        0  & 7 & a+2b
    \end{bmatrix}$$
$$\arrows1{\frac{1}{7}R_2}\begin{bmatrix}
        -1 & 3 & b        \\
        0  & 1 & (a+2b)/7
    \end{bmatrix}\arrows1{R_1-3R_2}\begin{bmatrix}
        -1 & 0 & (b-3a)/7 \\
        0  & 1 & (a+2b)/7
    \end{bmatrix}$$
$$x=\frac{3a-b}{7} \qquad \text{and} \qquad y=\frac{(a+2b)}{7}$$
Thus,
$$\frac{3a-b}{7}\begin{bmatrix}
        2 \\ -1
    \end{bmatrix}+\frac{a+2b}{7}\begin{bmatrix}
        1 \\ 3
    \end{bmatrix}=\begin{bmatrix}
        a \\ b
    \end{bmatrix}$$

A set of vectors $\*v_1,\*v_2,\dots,\*v_k$ is \textbf{linearly dependent} if there
are scalars $c_1,c_2,\dots,c_k$, at least one of which is not zero, such that
$$c_1\*v_1+c_2\*v_2+\dots+c_k\*v_k=0$$
A set of vectors that is not linearly dependent is called \textbf{linearly independent}.

\subsection*{Example}
Determine whether the following set of vectors are linearly independent:
$$\begin{bmatrix}
        1 \\ 2 \\ 0
    \end{bmatrix},\begin{bmatrix}
        1 \\ 1 \\ -1
    \end{bmatrix},\begin{bmatrix}
        1 \\ 4 \\ 2
    \end{bmatrix}$$

\subsection*{Solution}
\[
    \begin{bmatrix}
        1 & 1  & 1 & 0 \\
        2 & 1  & 4 & 0 \\
        0 & -1 & 2 & 0
    \end{bmatrix}\arrows1{R_2-2R_1}\begin{bmatrix}
        1 & 1  & 1 & 0 \\
        0 & -1 & 2 & 0 \\
        0 & -1 & 2 & 0
    \end{bmatrix}\arrows3{R_1+R_2}{R_3-R_2}{-R_2}\begin{bmatrix}
        1 & 0 & 3  & 0 \\
        0 & 1 & -2 & 0 \\
        0 & 0 & 0  & 0
    \end{bmatrix}
\]

\begin{enumerate}
    \item[] $c_1+3c_3=0 \qquad c_2-2c_3=0$
    \item[] $c_1=-3c_3 \qquad c_2=2c_3$
\end{enumerate}

\[
    -3c_3\begin{bmatrix}
        1 \\ 2 \\ 0
    \end{bmatrix}+2c_3\begin{bmatrix}
        1 \\ 1 \\ -1
    \end{bmatrix}+c_3\begin{bmatrix}
        1 \\ 4 \\ 2
    \end{bmatrix}=\begin{bmatrix}
        0 \\ 0 \\ 0
    \end{bmatrix}
\]

\section{Applications}

\subsection*{Example}
The combustion of ammonia ($\text{NH}_3$) in oxygen produces nitrogen
($\text{N}_2$) and water. Find a balanced chemical equation for this reaction.

\subsection*{Solution}
\[w{NH}_3+xO_2\to yN_2+zH_2O\]
\begin{enumerate}
    \item[] Nitrogen: $w=2y$
    \item[] Hydrogen: $3w=2z$
    \item[] Oxygen: $2x=z$
\end{enumerate}
\[
    \begin{matrix}
        w-2y=0  \\
        3w-2z=0 \\
        2x-z=0
    \end{matrix}\to\begin{bmatrix}
        1 & 0 & -2 & 0  & 0   \\
        3 & 0 & 0  & -2 & 0 & \\
        0 & 2 & 0  & -1 & 0
    \end{bmatrix}\arrows1{rref}\begin{bmatrix}
        1 & 0 & 0 & -2/3 & 0 \\
        0 & 1 & 0 & -1/2 & 0 \\
        0 & 0 & 1 & -1/3 & 0
    \end{bmatrix}
\]

\begin{align*}
    w=\frac{2}{3}z \qquad x=\frac{1}{2}z \qquad y=\frac{1}{3}z \\
    w=4 \qquad x=3 \qquad y=2 \qquad z=6                       \\
    4{NH}_3+3O_2\to2N_2+6H_2O
\end{align*}